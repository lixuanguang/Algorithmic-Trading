Based on the books by James Douglas \cite{hamilton_1994}, and ...

\subsubsection{Stationary Time Series}

\begin{definition}
A \hlt{linear first-order difference equation} is defined as
\begin{equation}
y_t = \phi y_{t-1} + w_t \nonumber
\end{equation}
where $y_t$ is the target variable (with $y_{t-1}$ the lag $1$ variable), $w_t$ is an input variable at time $t$
\end{definition}


The difference equation may be solved by recursive substitution to arrive at
\begin{equation}
y_t = \phi^{t+1}y_{-1} + \phi^{t} w_0 + \phi^{t-1} w_1 + \phi^{t-2} w_2 + \cdots + \phi w_{t-1} + w_t \nonumber
\end{equation}

The \hlt{dynamic multiplier} calculates effect of $w_t$ on $y_{t+j}$, and is given by
\begin{equation}
\frac{\partial y_{t+j}}{\partial w_t} = \phi^j \nonumber
\end{equation}
If $\abs{\phi} < 1$, the system is stable. If $\abs{\phi} > 1$, then the system is explosive.

We may generalise the process to \hlt{$p$-th order difference equation}, i.e.,
\begin{equation}
y_t = \phi_t y_{t-1} + \phi_2 t_{t-2} + \cdots + \phi_p y_{t-p} + w_t \nonumber
\end{equation}
We may rewrite this as a first-order difference equation in a ($p \times 1$) vector $\bm{\xi}_t$:
\begin{equation}
\bm{\xi}_t = 
\begin{bmatrix}
y_t \\
y_{t-1} \\ 
\vdots
y_{t-p+1}
\end{bmatrix} \nonumber
\end{equation}
Define the ($p \times p$) matrix $\bm{F}$ by
\begin{equation}
\bm{F} =
\begin{bmatrix}
\phi_1 & \phi_2 & \phi_3 & \cdots & \phi_{p-1} & \phi_p \\
1 & 0 & 0 & \cdots & 0 & 0 \\
0 & 1 & 0 & \cdots & 0 & 0 \\
\vdots & \vdots & \vdots & \ddots & \vdots & \vdots \\
0 & 0 & 0 & \cdots & 1 & 0
\end{bmatrix} \nonumber
\end{equation}
Finally, define the ($p \times 1$) vector $\bm{v}_t$ by
\begin{equation}
\bm{v}_t =
\begin{bmatrix}
w_t \\
0 \\
0 \\
\vdots \\
0
\end{bmatrix} \nonumber
\end{equation}
Then the system of $p$ equations $\bm{\xi}_t = \bm{F} \bm{\xi}_{t-1} + \bm{v}_t$ is identical to the $p$-th order difference equation.