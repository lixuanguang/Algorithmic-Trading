\usepackage[utf8]{inputenc}
\usepackage{fancyhdr}
\usepackage[margin=2cm,foot=1cm]{geometry}
\usepackage[T1]{fontenc}
\usepackage{hyperref}
\usepackage{ragged2e}
\usepackage{amsbsy}
\usepackage{xparse}
\usepackage{natbib}
\bibliographystyle{named}

\hypersetup{
	colorlinks,
	linkcolor={blue!80!black},
	citecolor={green!80!black},
	urlcolor={red!80!black}
}

\usepackage[nodisplayskipstretch]{setspace}
\setstretch{1}

% Used to set numbering of the content
\setcounter{tocdepth}{3}
\setcounter{secnumdepth}{3}

\newcommand{\matr}[1]{\mathbf{#1}} % undergraduate algebra version
%\newcommand{\matr}[1]{#1}          % pure math version
%\newcommand{\matr}[1]{\bm{#1}}     % ISO complying version
\newcommand{\vect}[1]{\bm{#1}} % undergraduate algebra version

%%
% Math packages
\usepackage[utf8]{inputenc}
\usepackage{mathtools}
\usepackage{amssymb}
\usepackage{bm}
\usepackage{amsthm}
\usepackage{amsmath}
\usepackage{rotating}
\usepackage{mathrsfs}
\usepackage{tikz-cd}
\usepackage{float}
\usepackage{enumitem}
\usepackage{makecell}

\graphicspath{ {./images/} }

%%
% Math Commands
\def\upint{\mathchoice%
	{\mkern13mu\overline{\vphantom{\intop}\mkern7mu}\mkern-20mu}%
	{\mkern7mu\overline{\vphantom{\intop}\mkern7mu}\mkern-14mu}%
	{\mkern7mu\overline{\vphantom{\intop}\mkern7mu}\mkern-14mu}%
	{\mkern7mu\overline{\vphantom{\intop}\mkern7mu}\mkern-14mu}%
	\int}
\def\lowint{\mkern3mu\underline{\vphantom{\intop}\mkern7mu}\mkern-10mu\int}
\usepackage{tikz}
\newcommand{\N}{\mathbb{N}}
\newcommand{\R}{\mathbb{R}}
\newcommand{\Q}{\mathbb{Q}}
\newcommand{\C}{\mathbb{C}}
\newcommand{\x}{\mathbf{x}}
\newcommand{\F}{\mathbf{F}}
\newcommand{\f}{\mathbf{f}}
\newcommand{\y}{\mathbf{y}}
\renewcommand{\b}{\mathbf{b}}
\renewcommand{\c}{\mathbf{c}}
\renewcommand{\a}{\mathbf{a}}
\newcommand{\h}{\mathbf{h}}
\newcommand{\g}{\mathbf{g}}
\newcommand{\z}{\mathbf{z}}
\newcommand{\ze}{\mathbf{0}}
\newcommand{\Z}{\mathbb{Z}}
\newcommand{\norm}[1]{\left\lVert#1\right\rVert}
\newcommand{\abs}[1]{\left\lvert#1\right\rvert}
\newcommand{\brk}[1]{ \left[#1\right] }
\newcommand{\brc}[1]{ \left\{#1\right\} }
\newcommand{\paren}[1]{ \left(#1\right) }
\newcommand{\normop}[1]{\left\lVert#1\right\rVert_\text{op}}
\newcommand{\LL}{\mathcal{L}}
\newcommand{\uni}{\overset{\text{uni}}{\to}}
\DeclareMathOperator{\diam}{diam}
\newcommand{\Prr}[1]{\text{Pr}\left(#1\right)}
\newcommand{\ceil}[1]{\lceil #1 \rceil}
\newcommand{\floor}[1]{\lfloor #1 \rfloor}

\definecolor{dartmouthgreen}{rgb}{0.05, 0.5, 0.06}
\definecolor{egyptianblue}{rgb}{0.06, 0.2, 0.65}
\definecolor{dukeblue}{rgb}{0.0, 0.0, 0.61}
\definecolor{jazzberryjam}{rgb}{0.65, 0.04, 0.37}
\definecolor{magenta}{HTML}{EC008C}
\definecolor{darkmagenta}{rgb}{0.55, 0.0, 0.55}
\definecolor{deeppink}{rgb}{1.0, 0.08, 0.58}

\newcommand\disteq{\stackrel{\textnormal{dist}}{=}}

\NewDocumentCommand{\evalat}{sO{\big}mm}{%
  \IfBooleanTF{#1}
   {\mleft. #3 \mright|_{#4}}
   {#3#2|_{#4}}%
}

%%
% Self-defined useful shortcuts
\newcommand{\isomorp}{\xrightarrow{\sim}}
\newcommand{\hlt}[1]{\textit{{\color{blue}#1}}}
\newcommand{\simpt}[1]{\textit{{\color{deeppink}#1}}}
\newcommand{\impt}[1]{\textit{{\color{red}#1}}}
\newcommand{\gcds}[1]{\textnormal{gcd}#1}
\newcommand{\mins}[1]{\textnormal{min}#1}
\newcommand{\maxs}[1]{\textnormal{max}#1}
\newcommand{\lcms}[1]{\textnormal{lcm}#1}
\newcommand{\degs}[1]{\textnormal{deg}#1}
\newcommand{\exps}[1]{\textnormal{exp}#1}
\newcommand{\tors}[1]{\textnormal{Tor}#1}
\newcommand{\homs}[1]{\textnormal{Hom}#1}
\newcommand{\anns}[1]{\textnormal{Ann}#1}
\DeclareMathOperator{\sign}{sign}

%%
\setcounter{section}{0}
% Set up math tools
\theoremstyle{theorem}
\newtheorem{theorem}{Theorem}[subsection]
\newtheorem{corollary}[theorem]{Corollary}
\newtheorem{lemma}[theorem]{Lemma}
\newtheorem{proposition}[theorem]{Proposition}
\newtheorem{qnbank}[theorem]{Question Bank}
\let\oldqnbank\qnbank
\renewcommand{\qnbank}{\oldqnbank\normalfont}
\newtheorem{algorithm}[theorem]{Algorithm}
\let\oldalgorithm\algorithm
\renewcommand{\algorithm}{\oldalgorithm\normalfont}
\newtheorem{process}[theorem]{Process}
\let\oldprocess\process
\renewcommand{\process}{\oldprocess\normalfont}
\newtheorem{method}[theorem]{Method}
\let\oldmethod\method
\renewcommand{\method}{\oldmethod\normalfont}
\newtheorem{definition}[theorem]{Definition}
\let\olddefinition\definition
\renewcommand{\definition}{\olddefinition\normalfont}
\newtheorem{example}[theorem]{Example}
\let\oldexample\example
\renewcommand{\example}{\oldexample\normalfont}
\newtheorem{remark}[theorem]{Remark}
\let\oldremark\remark
\renewcommand{\remark}{\oldremark\normalfont}

%% For coding syntax
% Default fixed font does not support bold face
\DeclareFixedFont{\ttb}{T1}{txtt}{bx}{n}{12} % for bold
\DeclareFixedFont{\ttm}{T1}{txtt}{m}{n}{12}  % for normal

% Custom colors
\usepackage{color}
\definecolor{deepblue}{rgb}{0,0,0.5}
\definecolor{deepred}{rgb}{0.6,0,0}
\definecolor{deepgreen}{rgb}{0,0.5,0}

\usepackage{listings}

% Python style for highlighting
\newcommand\pythonstyle{\lstset{
language=Python,
basicstyle=\ttm,
morekeywords={self, def, print},              % Add keywords here
keywordstyle=\ttb\color{deepblue},
emph={__init__},          % Custom highlighting
emphstyle=\ttb\color{deepred},    % Custom highlighting style
stringstyle=\color{deepgreen},
frame=tb,                         % Any extra options here
showstringspaces=false
}}

% Python environment
\lstnewenvironment{python}[1][]
{\pythonstyle
\lstset{#1}}{}

% Python for external files
\newcommand\pythonexternal[2][]{{
\pythonstyle
\lstinputlisting[#1]{#2}}}

% Python for inline
\newcommand\pythoninline[1]{{\pythonstyle\lstinline!#1!}}
