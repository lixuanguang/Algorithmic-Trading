\subsubsection{Risk Assessment}

\begin{definition} \hlt{Model Risks}\\
Quant models has model risk, the risk that the model does not accurately describe, match, or predict the real-world phenomenon. Each component of the quant model may all have model risk.
\begin{enumerate}[label=\roman*.]
\setlength{\itemsep}{0pt}
\item Inapplicability of Modelling: occurs when quant model is mistakenly applied to a problem. May also occur with misapplication of a technique to a given problem.
\item Model Misspecification: occurs when the model doesn't fit the real world. Model may work fine most of the time, but fail when an extreme event occurs.
\item Implementation Errors: erors in programming or architecting systems. Architectural error may also occur when models are loaded in a wrong sequence.
\end{enumerate}
\end{definition}

\begin{definition} \hlt{Regime Change Risk}\\
Quant models are based on relationships prevalent in historical data. If there is a regime change, the historical relationships and behaviour may be altered, hence the model may lose effectiveness.
\end{definition}

\begin{definition} \hlt{Exogenous Shock Risk}\\	
Risks driven by information that is not internal to the market, i.e., terrorist attacks, start of wars, bank bailouts, change in regulation such as in shorting rules. May require discretionary overrides.
\end{definition}

\begin{definition} \hlt{Contagion Risk}\\
Happens when other investors hold the same strategies. First part of risk factor relates to how crowded the quant strategy is. Second part relates to what else is held by other investors that could force them to exit the quant strategy in a panic (ATM effect).\\
Quant liquidation criss may be driven by size and popularity of quantitative strategies, subpar returns from operators leading up to the crisis, the practice of funds cross-collateralising many strategies against each other, and risk targeting (risk managers target a specific level of volatility for their funds or strategies).
\end{definition}

\begin{method} \hlt{Risk Monitoring Methods}
\begin{enumerate}[label=\roman*.]
\setlength{\itemsep}{0pt}
\item Exposure Monitoring Tools: with current positions held, the positions are grouped for the various exposures (i.e., valuation, momentum level, volatility) to monitor gross and net exposure to various sectors and industries, buckets of market capitalisation, various style factors.
\item Profit and Loss Monitors: with current portfolio, compare that with previous day closing price. Intraday performance charts are used. May also look at source of profit, hit rate (percentage of time strategy makes money on a given position).
\item Execution Monitors: shows progress of executions, i.e., which orders are currently being worked on, which ones are completed, with transaction size and prices. Fill rates for limit orders are used for more passive execution strategies. Slippage and market impact are also monitored.
\item System Performance Monitors: checks for software and infrastructure errors. Checks performance of CPUs, speeds of various stages of automated processes, latency in communication of messages.
\end{enumerate}
\end{method}