\subsubsection{Data Taxonomy}

A brief overview of the types of data used in systematic trading.

\begin{flushleft}
Four essential types of financial data
\begin{tabularx}{\textwidth}{X|X|p{13em}|X}
\hline
\rowcolor{gray!30}
Fundamental Data & Market Data & Analytics & Alternative Data \\
\hline
Assets \newline
Liabilities \newline
Sales \newline
Costs/Earnings \newline
Macro Variables \newline
$\cdots$
&
Price/Yield/IV \newline
Volume \newline
Dividend/Coupons \newline
Open Interest \newline
Quotes/Cancellations \newline
Aggressor Side \newline
$\cdots$
&
Analyst Recommendation \newline
Credit Ratings \newline
Earnings Expectations \newline
News Sentiment \newline
$\cdots$
&
Satellite/CCTV \newline
Google Searches \newline
Twitter/Chats \newline
Metadata \newline
$\cdots$ \\
\hline
\end{tabularx}
\end{flushleft}

\begin{remark} \hlt{Fundamental Data Characteristics}
\begin{enumerate}[label=\roman*.]
\setlength{\itemsep}{0pt}
\item Data is published with an index corresponding to last date in the report, which precedes the release date.
\item Data is frequently backfilled or corrected, with the data vendor overwriting initial values as needed.
\item The data is highly regularized and available at low frequency.
\end{enumerate}
\end{remark}

\begin{remark} \hlt{Market Data Characteristics}
\begin{enumerate}[label=\roman*.]
\setlength{\itemsep}{0pt}
\item The raw feed consists of unstructured information, such as FIX messages (which allow full reconstruction of the trading book) or complete collections of BWIC (bids wanted in competition) responses.
\item Processing FIX data is non-trivial, with approximately $10$TB generated daily.
\end{enumerate}
\end{remark}

\begin{remark} \hlt{Analytics Data Characteristics}
\begin{enumerate}[label=\roman*.]
\setlength{\itemsep}{0pt}
\item This is derivative data processed from the original source, with the relevant signal already extracted.
\item It is costly to produce, and the methodology used in production may be biased or opaque.
\end{enumerate}
\end{remark}

\begin{remark} \hlt{Alternative Data Characteristics}
\begin{enumerate}[label=\roman*.]
\setlength{\itemsep}{0pt}
\item This data is generated by individuals, business processes, and sensors.
\item It provides primary information that is not available from traditional sources.
\item Cost and privacy concerns; it may be particularly valuable if it challenges existing data infrastructure.
\end{enumerate}
\end{remark}

\begin{definition} \hlt{Reference Data}
\begin{enumerate}[label=\roman*.]
\setlength{\itemsep}{0pt}
\item \textbf{Trading Universe:} Evolves daily to incorporate new listings and de-listings. Knowing when a stock ceases trading is crucial to avoid survivor bias.
\item \textbf{Symbology Mapping:} Involves identifiers such as ISIN, SEDOL, RIC, and Bloomberg Tickers. Since symbols may change over time, mapping must persist as point-in-time data to support historical 'as-of-date' analyses, often requiring a bi-temporal data structure.
\item \textbf{Ticker Changes:} Maintain a historical table of ticker changes (as described in symbology mapping) to ensure seamless continuity in time series data.
\item \textbf{Corporate Actions Calendars:} Include events such as stock and cash dividends (with announcement and execution dates), stock splits, reverse splits, rights offers, mergers and acquisitions, spin-offs, adjustments in free float or shares outstanding, and quotation suspensions.\\
For dividends, announcements may coincide with increased volatility and price jumps, enabling strategies to capitalize on the added volatility.For splits and rights offers, historical data must be adjusted backward (both volume and price) to reflect these actions. For M\&A and spin-offs, adjustments are needed to account for valuation changes, which are important in merger arbitrage strategies. Suspensions can create data gaps that impact backtesting.
\item \textbf{Static Data:} Comprises attributes like country, sector, primary exchange, currency, and quote factor. This data is used to group instruments based on fundamental similarities (e.g., for pairs trading), and maintaining a table of quotation currencies per instrument is essential for portfolio aggregation.
\item \textbf{Exchange Specific Data:} Each exchange has unique features that must be considered. \\
\underline{First Group}: Hours and Dates of Operations
\begin{enumerate}[label=\arabic*.]
\setlength{\itemsep}{0pt}
\item \textbf{Holiday Calendar:} Different markets have their own holiday schedules. For strategies that trade across multiple markets, discrepancies in holiday closures can affect correlation.
\item \textbf{Exchange Session Hours:} Different sessions (Pre-Market, Continuous Core, After-Hours, etc.), auction times, cutoff times for order submission, lunch breaks, and pre-/post-lunch auctions. This also includes DST adjustments and variations in trading hours by venue.
\item \textbf{Disrupted Days:} Records of exchange outages or trading disruptions, which are important to filter out when building or testing strategies.
\end{enumerate}
\underline{Second Group}: Trading Mechanics
\begin{enumerate}[label=\arabic*.]
\setlength{\itemsep}{0pt}
\item \textbf{Tick Size:} The minimum eligible price increment, which may vary by instrument and price level.
\item \textbf{Trade and Quote Lots:} The minimum size increments for trades or quotes.
\item \textbf{Limit-Up and Limit-Down Constraints:} Maximum daily price fluctuations and the rules for trading pauses or restrictions at threshold levels.
\item \textbf{Short Sell Restrictions:} Rules that may prevent short sales from trading at prices worse than the last trade, or from generating new quotes below the lowest prevailing price. These restrictions impact liquidity sourcing.
\end{enumerate}
\item \textbf{Market Data Condition Codes:} Vary by exchange and asset class. Each market event may have multiple codes (e.g., auction trade, lit/dark trade, cancelled/corrected trade, regular trade, off-exchange reporting, block trade, or multi-leg order such as an option spread). It is essential to build a mapping table for these codes so that trades published solely for reporting purposes can be excluded from liquidity updates in aggregated daily data.
\item \textbf{Special Day Calendars:} Identify days with distinct liquidity characteristics that impact both execution strategies and alpha generation. Examples (non-exhaustive) include:
\begin{enumerate}[label=\arabic*.]
\setlength{\itemsep}{0pt}
\item Half trading days before Christmas and after Thanksgiving (US).
\item Ramadan effects in Turkey.
\item Taiwan market opening on a weekend to compensate for holiday closures.
\item Adjusted trading hours in Korea on nationwide university entrance exam days.
\item Late openings in the Brazilian market following Carnival.
\item Last trading days of months and quarters, when portfolio rebalancing occurs.
\item Index rebalancing dates, where intraday volume skews toward the end of day.
\item Options and futures expiry dates (e.g., quarterly/monthly expiry, Triple Witching in the US, Special Quotations in Japan) that cause excess trading volume and altered intraday patterns due to hedging and portfolio adjustments.
\end{enumerate}
Model normal days first; special days either modelled independently or by adjusting normal day baseline.
\item \textbf{Futures-Specific Reference Data:} Essential for determining which contract was live at any point via an expiry calendar and identifying the most liquid contract. For example, equity index futures are generally most liquid for the front month, while energy futures (such as oil) might be more liquid in the second contract. Note that there is no standardised expiry frequency across markets. When computing rolling-window metrics, potential roll dates must be accounted for; volume data may be blended before and after a roll. Additionally, futures markets exhibit different intra-day phases with distinct liquidity characteristics, so market data metrics (volume profiles, average spreads, bid-ask sizes) should be computed for each session based on a schedule.
\item \textbf{Options-Specific Reference Data (Options Chain):} Consists of expiry date and strike price combinations. Mapping equity tickers to their corresponding option tickers, with strike and expiry information, facilitates more complex investment and hedging strategies (e.g., assessing distance to strike or changes in open interest between puts and calls).
\item \textbf{Market-Moving News Releases:} Includes macroeconomic announcements (central bank statements such as FED/FOMC, ECB, BOE, BOJ, SNB; Non-Farm Payrolls; PMIs; Manufacturing Indices; Crude Oil Inventories) and stock-specific events like earnings releases. Maintaining a calendar of these events helps assess their impact on strategies.
\item \textbf{Related Tickers:} Tickers that represent the same underlying asset, allowing for efficient opportunity identification. This includes mappings for primary tickers to composite tickers (in fragmented liquidity markets), dual-listed or fungible securities (e.g., in the US and Canada), ADRs/GDRs, or differences between local and foreign boards.
\item \textbf{Composite Assets:} Such as ETFs, indexes, and mutual funds, which are used to achieve desired exposures or as hedging instruments. They can also present arbitrage opportunities when deviating from their NAV. It is important to maintain data on their constituent time series, any cash component, the divisor for converting NAV to quoted price, and the constituent weights.
\item \textbf{Latency Tables:} For high-frequency trading, these tables record the distribution of latencies between different data centers for efficient order routing, as well as reordering data collected from different locations.
\end{enumerate}
\end{definition}

\begin{definition} \hlt{Market Data Feed}
\begin{enumerate}[label=\roman*.]
\setlength{\itemsep}{0pt}
\item \textbf{Level I Data (Trade and BBO Quotes):} Contains trade executions and top-of-book quotes, sufficient to reconstruct the Best Bid and Offer (BBO). This data also includes trade status (e.g., cancelled, reported late) and qualifiers (e.g., odd lot, normal trade, auction trade, intermarket sweep, average price reporting, exchange details), which are useful for analyzing event sequences and deciding whether a print should update the last price and total traded volume.
\item \textbf{Level II Data (Market Depth):} Provides full depth of the limit order book, including all updates (price changes, additions, or removals of shares) across all venues in fragmented markets.
\item \textbf{Level III Data (Full Order View):} Provides detailed message data in which each incoming order is assigned a unique ID for tracking. Records precise details when an order is executed, cancelled, or amended, making it possible to reconstruct complete order book (with national depth) at any moment.
\begin{enumerate}[label=\arabic*.]
\setlength{\itemsep}{0pt}
\item \textbf{Timestamp:} Milliseconds elapsed since midnight.
\item \textbf{Ticker:} Equity symbol (up to 8 characters).
\item \textbf{Order:} Unique order identifier.
\item \textbf{T (Message Type):} 'B' indicates an add buy order; 'S' an add sell order; 'E' a partial execution; 'C' a partial cancellation; 'F' a full execution; 'D' a full deletion; 'X' a bulk volume for a cross event; and 'T' an execution of a non-displayed order.
\item \textbf{Shares:} Order quantity for messages 'B', 'S', 'E', 'X', 'C', and 'T'. Zero for 'F' and 'D'.
\item \textbf{Price:} Order price for 'B', 'S', 'X', and 'T' messages; zero for cancellations and executions. The last 4 digits denote the decimal part (padded with zeroes), and the value should be divided by 1000 to convert to the currency unit.
\item \textbf{MPID:} A 4-character Market Participant ID associated with the transaction.
\item \textbf{MCID:} A 1-character Market Centre Code for the originating exchange.
\end{enumerate}
\item Special order types of note:
\begin{enumerate}[label=\arabic*.]
\setlength{\itemsep}{0pt}
\item \textbf{Order Subject to Price Sliding:} Execution price may be one cent less favourable than display price (e.g., at NASDAQ). Such orders are ranked at locking price as hidden orders but displayed at one minimum price variation inferior; a new order ID is issued if order is replaced as a display order.
\item \textbf{Pegged Order:} Based on the NBBO, these orders are non-routable and receive a new timestamp upon repricing; display rules vary by exchange.
\item \textbf{Mid-point Peg Order:} A non-displayed order that may result in half-penny executions.
\item \textbf{Reserve Order:} The displayed size is treated as a displayed limit order, while the reserve size is subordinate to non-displayed and pegged orders. The minimum display quantity is 100 shares; when it falls below this threshold, the reserve is replenished, a new timestamp is generated, and the displayed size is re-ranked.
\item \textbf{Discretionary Order:} Displayed at one price while passively trading at a more aggressive discretionary price. It only becomes active when shares are available within the discretionary price range and is ranked last in priority. The execution price may be less favourable than the display price.
\item \textbf{Intermarket Sweep Order:} Can be executed without the need to verify the prevailing NBBO.
\end{enumerate}
Using this comprehensive Level III data, one can model the inter-arrival times of various events, as well as the arrival and cancellation rates as functions of distance from the best bid/offer and other variables (e.g., order book imbalance, queue length). Subsequently, analysis may include assessing the impact of market orders on the limit order book, estimating the likelihood of a limit order advancing in the queue, determining the probability of capturing the spread, and forecasting short-term price movements.
\end{enumerate}
\end{definition}

\begin{definition} \hlt{Binned Data}
\begin{enumerate}[label=\roman*.]
\setlength{\itemsep}{0pt}
\item \textbf{Open, High, Low, Close (OHLC) and Previous Close Price:} These values indicate trading activity and intraday volatility. The range between the low and high prices reflects market sentiment, and the previous close must be adjusted for corporate actions and dividends.
\item \textbf{Last Trade before Close (Price/Size/Time):} Captures any jump in the close price during the final trading moments, serving as an indicator of the stability of the close as a reference for the next day.
\item \textbf{Volume:} Acts as an indicator of trading activity, particularly when it deviates sharply from long-term averages. It is useful to collect volume breakdowns between lit and dark venues for execution strategies.
\item \textbf{Auctions Volume and Price:} Reflects price discovery events marked by significant volume prints.
\item \textbf{VWAP:} The Volume Weighted Average Price offers an indication of daily trading activity and is particularly useful for algorithmically executing large orders.
\item \textbf{Short Interest/Days-to-Cover/Utilisation:} These metrics serve as proxies for investor positioning. High short interest suggests a bearish view from institutional investors, while the utilisation of borrowable securities indicates the potential for additional shorting. Days-to-Cover helps assess the potential severity of a short squeeze; higher values imply a greater chance of sudden price surges in heavily shorted securities.
\item \textbf{Futures Data:} Provides insights into market activity or positions of large investors through open interest data. Arbitrage opportunities may arise if the basis is mispriced relative to dividend estimates.
\item \textbf{Index-Level Data:} Supplies relative measures for instrument-specific features (such as index OHLC and volatility). Normalised features can help identify instruments that deviate from their benchmarks.
\item \textbf{Options Data:} Offers information on trader positioning via open interest and the Greeks.
\item \textbf{Asset Class Specific Data:} Includes yield or benchmark rates (such as repo rates, 2-year, 10-year, and 30-year yields), CDS spreads, and the US Dollar Index.
\end{enumerate}
\end{definition}

\begin{definition} \hlt{Granular Intraday Microstructure Activity}
\begin{enumerate}[label=\roman*.]
\setlength{\itemsep}{0pt}
\item \textbf{Number and Frequency of Trades:} Serves as a proxy for market activity and continuity; a low number may indicate challenging execution and higher volatility.
\item \textbf{Number and Frequency of Quote Updates:} Provides a similar measure of market activity.
\item \textbf{Top of Book Size:} Liquidity; larger sizes allow for larger orders to be executed almost immediately.
\item \textbf{Depth of Book (Price and Size):} Also reflects the liquidity available in the market.
\item \textbf{Spread Size (Average, Median, Time-Weighted Average):} Proxy for trading costs. Parameterised distribution of spreads helps in identifying when trading opportunities are relatively inexpensive or costly.
\item \textbf{Trade Size (Average, Median):} Useful for identifying intraday liquidity opportunities by examining the volume available in the order book.
\item \textbf{Ticking Time (Average, Median):} Represents how frequently the top level of the order book is updated. This measure is critical for execution algorithms that must adapt their update frequency (e.g., for adding or cancelling child orders) to the characteristics of the traded instrument.
\end{enumerate}
Daily distributions of these metrics can be used as starting estimates at the beginning of the day and updated intraday via online Bayesian methods.\\
Derived daily data include:
\begin{enumerate}[label=\roman*.]
\setlength{\itemsep}{0pt}
\item X-day Average Daily Volume (ADV) / Average Auction Volume
\item X-day Volatility (e.g., close-to-close, open-to-close)
\item Beta with respect to an index or sector (using standard beta or asymmetric up-/down-day beta)
\item Correlation Matrix
\end{enumerate}
When binning data, intervals may range from a few seconds to 30 minutes. Minute-bar data is typically used for volume and spread profiles to reduce noise from market friction.
\end{definition}

\begin{definition} \hlt{Fundamental Data and Other Data}
\begin{enumerate}[label=\roman*.]
\setlength{\itemsep}{0pt}
\item \textbf{Key Ratios:} Such as Earnings Per Share (EPS), Price-to-Earnings (P/E), Price-to-Book (P/B), etc.
\item \textbf{Analyst Recommendations:} Aggregated consensus valuations from analysts.
\item \textbf{Earnings Data:} Quarterly earnings estimates provided by research analysts serve as indicators of a stock's performance prior to the actual published figures.
\item \textbf{Holders:} Significant changes in ownership can indicate shifts in sentiment among sophisticated investors.
\item \textbf{Insider Purchase/Sale:} Reflects potential future price movements, as insiders typically possess the best available information about the company.
\item \textbf{Credit Ratings:} Downgrades, which lead to higher funding costs, can adversely impact equity prices.
\end{enumerate}
\end{definition}