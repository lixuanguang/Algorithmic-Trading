\subsubsection{Data Labelling Techniques}

\begin{method} \hlt{Labelling with Fixed-Time Horizon Method}\\
Consider a features matrix $X$ composed of $I$ rows, where each observation $\{X_i\}_{i=1,\ldots,I}$ is drawn from a series of bars indexed by $t = 1,\ldots,T$ (with $I \leq T$). In this method, each observation $X_i$ is assigned a label $y_i \in \{-1, 0, 1\}$ according to the following rule:
\begin{align}
y_i &=
\begin{cases}
-1, & \text{if } r_{t_{i,0}, t_{i,0} + h} < -\tau, \\
0,  & \text{if } \abs{r_{t_{i,0}, t_{i,0} + h}} \leq \tau, \\
1,  & \text{if } r_{t_{i,0}, t_{i,0} + h} > \tau,
\end{cases} \nonumber \\
r_{t_{i,0}, t_{i,0} + h} &= \frac{p_{t_{i,0}+h}}{p_{t_{i,0}}} - 1 \nonumber
\end{align}
Here, $\tau$ is a predetermined constant threshold, $t_{i,0}$ represents the index of the bar immediately after the occurrence of $X_i$, and $t_{i,0}+h$ denotes the $h$-th bar following $t_{i,0}$.
\end{method}

\begin{remark} \hlt{Limitations of Fixed-Time Horizon Method}
\begin{enumerate}[label=\roman*.]
\setlength{\itemsep}{0pt}
\item Time bars do not exhibit desirable statistical properties (as discussed previously).
\item The same threshold $\tau$ is uniformly applied regardless of the prevailing volatility. It is often beneficial to compute daily volatility at intraday estimation points by applying an exponentially weighted moving standard deviation over a span of $n$ days.
\end{enumerate}
\end{remark}

\begin{method} \hlt{Labelling with Triple-Barrier Method}\\
This method assigns a label to an observation based on the first barrier touched among three defined barriers.
\begin{enumerate}[label=\roman*.]
\setlength{\itemsep}{0pt}
\item Establish two horizontal barriers and one vertical barrier. Horizontal barriers are defined by profit-taking and stop-loss limits, dynamically determined as a function of estimated volatility (either realised or implied). The vertical barrier represents the expiration limit in terms of the number of bars elapsed since the position was taken.
\item If the upper barrier is touched first, label the observation as $1$. Conversely, if the lower barrier is reached first, label it as $-1$. If the vertical barrier is touched first, label the observation either by the sign of the return or with $0$.
\end{enumerate}
Note that this method is path-dependent. To label an observation, one must consider the entire price path over the interval $[t_{i,0}, t_{i,0} + h]$, where $h$ defines the vertical barrier (expiration limit). Let $t_{i,1}$ be the time at which first barrier is touched, with the associated return given by $r_{t_{i,0}, t_{i,1}}$. horizontal barriers may not be symmetric.
\end{method}

\begin{remark} \hlt{Triple-Barrier Method Configurations}\\
A barrier configuration is denoted by the triplet $[pt, sl, t1]$, which represents the upper barrier, lower barrier, and the physical (vertical) barrier, respectively. A value of $0$ indicates that the barrier is inactive, while a value of $1$ signifies an active barrier. The three useful configurations are:
\begin{enumerate}[label=\roman*.]
\setlength{\itemsep}{0pt}
\item $[1,1,1]$: To capture profit while imposing a maximum tolerance for losses and a defined holding period.
\item $[0,1,1]$: To exit after a set number of bars unless a stop-loss is triggered.
\item $[1,1,0]$: To secure profit as long as the stop-loss is not hit.
\end{enumerate}
The three less realistic configurations are:
\begin{enumerate}[label=\roman*.]
\setlength{\itemsep}{0pt}
\item $[0,0,1]$: Essentially equivalent to the fixed-time horizon method.
\item $[1,0,1]$: Position held until a profit is achieved or the maximum holding period is exceeded, ignoring any immediate unrealised losses.
\item $[1,0,0]$: Position maintained until a profit is made, potentially resulting in a very prolonged holding period.
\end{enumerate}
The two illogical configurations are:
\begin{enumerate}[label=\roman*.]
\setlength{\itemsep}{0pt}
\item $[0,1,0]$: This configuration is ambiguous, as it holds the position solely until a stop-loss occurs.
\item $[0,0,0]$: With no barriers in place, the position remains open indefinitely, and no label can be generated.
\end{enumerate}
\end{remark}

\begin{figure}[H]
\centering
\scalebox{0.75}{\input{figures/intro/metalabel.tikz}}
\caption{Meta-Labelling Process}
\end{figure}

\begin{method} \hlt{Meta-Labelling}\\
Meta-labelling is a technique that is particularly effective for achieving higher F1-scores.\\
The process begins by constructing a primary model that prioritizes high recall, even if its precision is not optimal. Subsequently, meta-labelling is applied to the positive predictions made by the primary model in order to filter out false positives. In essence, the secondary model is designed to discern whether a positive prediction from the primary model is truly valid.
\begin{enumerate}[label=\roman*.]
\setlength{\itemsep}{0pt}
\item Train a primary binary classification model.
\item Determine a threshold at which the primary model attains high recall; ROC curves can be employed to aid in selecting an appropriate threshold.
\item Construct a secondary model using features that typically include:
\begin{enumerate}[label=\roman*.]
\setlength{\itemsep}{0pt}
\item The primary model’s features concatenated with its predictions.
\item Indicators of the current market state.
\item Features that signal potential false positives.
\item Distribution-related characteristics.
\item Recent performance metrics of the primary model.
\end{enumerate}
The meta labels serve as the target variable for this secondary model.
\item The final prediction is made by combining the outputs of both models—only when both the primary and secondary models predict a positive does the observation receive a true positive label.
\end{enumerate}
\end{method}

\begin{remark} \hlt{Limitations of Meta-Labelling}
\begin{enumerate}[label=\roman*.]
\setlength{\itemsep}{0pt}
\item If the primary model overfits the data, the addition of meta-labelling may provide little to no benefit.
\item When trades are not considered independent observations, the meta-model may inadvertently be forced to capture day-to-day exposures, which is not the intended application of the technique.
\item This technique involves trading recall for precision; it requires a large number of trades for effective training, while accepting a reduced frequency of trades.
\end{enumerate}
\end{remark}

