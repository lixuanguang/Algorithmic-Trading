\subsubsection{Data Sample Weights}

Note that most of ML literature is based on IID assumption, and ML applications usually fail in finance as these assumptions are unrealistic in the case of financial time series.

\begin{remark} \hlt{Overlapping Outcomes}\\
Let label $y_i$ be assigned to an observed feature $X_i$, where $y_i = f([t_{i,0}, t_{i,1}])$ is a function over the interval. When $t_{i,1} > t_{j,0}$ and $i < j$, then $y_j$ will depend on common return $r_{t_{j,0}, \min\{t_{i,1}, t_{j,1}\}}$ (over interval $[t_{j,0}, \min\{t_{i,1}, t_{j,1}\}]$). The series of labels $\{y_i \}_{i-1, \ldots, J}$ are not IID whenever there is overlap between any two consecutive outcomes, i.e., $\exists i \ \vert \ t_{i,1} > t_{i+1, 0}$. If this is resolved by restricting bet horizon to $t_{i,1} \leq t_{i+1, 0}$, there is no overlap, but this will lead to coarse models where features sampling frequency is limited by horizon used to determine outcome.\\
To investigate outcomes that lasted a different duration, samples have to be resampled with different frequency. In addition, if path-dependent labelling technique is to be applied, the sampling frequency will be subordinated to first barrier's touch. Hence, to use $t_{i,1} > t_{i+1, 0}$, leading to overlapping outcomes.
\end{remark}