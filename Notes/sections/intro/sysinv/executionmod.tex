\subsubsection{Execution Model}

There are two primary methods to execute a trade: electronically or via a human intermediary. In electronic execution, direct market access (DMA) is employed, allowing traders to leverage the brokerage firm’s infrastructure and exchange connectivity to trade directly on electronic markets.\\
Execution algorithms can be developed in-house, sourced from brokers, or obtained from third-party vendors.\\
Brokerages also provide portfolio bidding services. In these arrangements, the “blind” portfolio for the transaction is characterized by features such as valuation ratios of long and short positions, sector allocation, market capitalization, and similar metrics. The broker then quotes a fee—expressed in basis points relative to the gross market value of the portfolio traded—which offers the trader a measure of certainty. Once an agreement is reached, the broker collects the fee and assumes the risk of executing the portfolio at future market prices, which may turn out to be more or less favorable than the initially guaranteed prices..

\begin{remark} \hlt{Order Execution Algorithm Parameters}
\begin{enumerate}[label=\roman*.] \setlength{\itemsep}{0pt}

\item \textbf{Aggressive vs Passive:} Algorithm decides whether to use an aggressive or passive order based on how immediately the trade must be executed. Market orders are inherently aggressive. Limit order placed at current best quote is relatively aggressive; limit order positioned below current bid is considered passive.\\
Many exchanges reward liquidity providers for placing passive orders, while charging traders for consuming liquidity. When an order crosses the spread, it effectively uses liquidity from another trader’s passive order, thereby reducing available liquidity. In return, if the passive order is executed, the trader can benefit from a better transaction price along with a commission rebate from the exchange.\\
Momentum-based strategies typically favour aggressive orders, whereas mean reversion strategies lean towards passive orders. A strong, reliable signal justifies a more aggressive execution, while a weaker or less certain signal may call for a more passive approach. An intermediate strategy might involve placing limit orders between the best current bid and offer.
\item \textbf{Large vs Small Order:} Large order may be divided into several smaller orders to reduce market impact, though this carries risk of adverse price movements. The optimal size of each order segment is determined by estimates from a transaction cost model and an analysis of the appropriate level of aggressiveness.
\item \textbf{Hidden vs Visible Order:} Visible order discloses certain trading intentions, whereas a hidden order conceals this information, helping to prevent market imbalances; however, hidden orders typically suffer from lower execution priority. In algorithmic trading, the practice of using hidden orders—where a large order is segmented into many smaller parts, with most being placed as hidden—is known as “iceberging.”
\item \textbf{Order Routing:} When multiple liquidity pools exist for same instrument, smart order routing systems are used to determine most suitable pool for executing a given order. These systems evaluate factors such as depth of liquidity on various electronic communication networks (ECNs) and connectivity speeds.
\item \textbf{Cancelling and Replacing Orders:} Traders may submit a large number of orders without the expectation of execution, only to rapidly cancel and replace them. This practice helps gauge the market’s response to changes in order book depth, offering insights into potential profitable patterns. For example, if a trader intends to buy a significant number of shares, they might place numerous small orders at prices further from the market and then cancel them, thereby improving market perception.
\end{enumerate}
\end{remark}

\begin{definition} \hlt{High Frequency Trading}\\
High Frequency Trading (HFT) involves alpha-driven strategies that focus on extremely short-term bets—often executed in seconds or less—referred to as \hlt{microstructure alphas}. These strategies analyze liquidity patterns in the order book. Larger quantitative groups may also incorporate these insights into their execution models to reduce the costs associated with entering trades. Even marginal improvements per trade can accumulate significantly over time. However, pursuing microstructure alpha as a standalone high frequency strategy necessitates substantial investments in both infrastructure and research.\\
Machine learning techniques may be applied to detect patterns in how other market participants execute orders. When competitors’ execution models are less refined, their patterns become more apparent, allowing an ML strategy to exploit these patterns in the future. Short-term patterns tend to exhibit a degree of stability.
\end{definition}

\begin{definition} \hlt{HFT Shark Strategy}\\
This is designed to detect large orders that have been fragmented (or “iceberged”) by observing the rapid filling of a series of very small trades. Quick execution of these small orders can indicate the presence of a large hidden order. The strategy then involves front-running this iceberg order by placing visible trades ahead of it. Consequently, the iceberg order is forced to move market prices upward in order to execute its trades. Once the iceberg order is fully executed, the resulting favorable price movement enables the shark to exit its position quickly, thereby securing a relatively risk-free profit.
\end{definition}

\begin{remark} \hlt{HFT Trading Infrastructure}\\
By using a broker that acts as a trading agent, traders can offload infrastructure requirements and avoid dealing directly with regulatory and other operational constraints.
High frequency strategies may also employ colocation or sponsored access. In a colocation setup, traders place their trading servers in close physical proximity to the exchange to minimize latency.\\
The Financial Information eXchange (FIX) protocol is the standard for real-time electronic communication among market participants. Although open source FIX engines are available, high frequency traders often develop their own proprietary FIX engines to ensure optimal execution speeds.
\end{remark}

