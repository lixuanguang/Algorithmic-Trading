\subsubsection{Alpha Models Overview}

Alpha models are designed to test and exploit theories about market behavior. These fall into two categories:
\begin{itemize}
  \item \textbf{Theory-Driven Models}, which derive signals from economic or behavioral principles, and
  \item \textbf{Data-Driven Models}, which rely on complex data mining and statistical techniques.
\end{itemize}
In practice, practitioners often blend several models together to capture multiple facets of market dynamics.

\begin{definition} \hlt{Theory-Driven Models}\\
Theory-driven models use fundamental principles to explain and predict market behavior. These include:
\begin{enumerate}[label=\roman*.]
  \setlength{\itemsep}{0pt}
  \item \textbf{Trend Following:}  
  Assumes that once a market trend is established, it will persist long enough to be identified and exploited. As supporting data accumulate for a bullish or bearish outlook, more participants join the trend, shifting the asset price to a new equilibrium. Note that strategies such as moving average crossovers often yield less than a one-to-one return relative to the downside risk because market behavior can be unstable and episodic.

  \item \textbf{Mean Reversion:}  
  Based on the idea that asset prices will revert to their historical average after deviating. Short-term liquidity imbalances can force prices to move abruptly, but these are typically followed by corrective movements. Statistical arbitrage, which bets on the convergence of prices between similar stocks that have diverged, is a common application of this strategy.

  \item \textbf{Value/Yield:}  
  Evaluates securities by comparing fundamental ratios to their market price (often using an inverted ratio for consistency). A higher yield indicates a relatively cheaper instrument. This approach underpins \hlt{carry trades}, where investors buy undervalued assets and sell overvalued ones. In \hlt{Quant Long Short (QLS)} strategies, stocks are ranked on factors like value, and long positions are taken in the most attractive stocks while shorting the least attractive.

  \item \textbf{Growth:}  
  Focuses on the future or historical growth potential of an asset. Forward-looking growth expectations are central, with the belief that companies with strong growth will increasingly dominate their sectors. Macro-level growth factors can drive foreign exchange decisions, while micro-level factors are key for individual companies.

  \item \textbf{Quality:}  
  Emphasizes capital preservation by favoring high-quality assets—those with robust earnings, sound balance sheets, and low leverage—over lower quality ones.
\end{enumerate}
\end{definition}

While theory-driven models incorporate clear economic rationale, \textbf{data-driven models} use advanced statistical and machine learning methods to extract patterns directly from data. These techniques, though more mathematically complex and often applied in high-frequency environments, can identify subtle market signals without relying on traditional economic theory.

\begin{method} \hlt{Strategy Parameters}\\
Implementing an alpha strategy requires careful specification of several parameters:
\begin{enumerate}[label=\roman*.]
  \setlength{\itemsep}{0pt}
  \item \textbf{Forecast Target:}  
  Define what the model predicts – whether it is the direction, magnitude, duration of a move, or even the probability of a particular outcome. A stronger signal (in terms of higher expected return or likelihood) typically warrants a larger position.

  \item \textbf{Time Horizon:}  
  Forecast horizons can range from microseconds to years. Short-term strategies usually involve a high frequency of trades, whereas long-term strategies trade less frequently.

  \item \textbf{Bet Structure:}  
  Models may generate absolute forecasts or relative forecasts (comparing one instrument to another). For relative forecasts, grouping assets into pairs or clusters (e.g., sectors) is common. While pair trading allows for precise comparisons, larger groupings can isolate idiosyncratic movements from overall market trends. Grouping can be achieved through statistical methods or heuristic industry classifications, each with its own tradeoffs.

  \item \textbf{Investment Universe:}  
  Selection is based on factors such as geography, asset class, and instrument type. High liquidity is essential for reliable transaction cost estimation and consistent behavior. Consequently, common stocks, futures (on bonds and equity indices), and forex are frequently modeled, while more volatile instruments (e.g., certain biotech stocks) are typically excluded.

  \item \textbf{Model Specification:}  
  This defines the mathematical structure of the strategy, including any machine learning or data mining techniques used. Regular model refitting is necessary to adapt to changing market conditions, although it may increase the risk of overfitting.

  \item \textbf{Run Frequency:}  
  This parameter sets how often the model is executed, ranging from monthly to real time. A higher run frequency increases transaction costs and sensitivity to noise, whereas a lower frequency can lead to larger trades that might themselves influence market prices.
\end{enumerate}
\end{method}

\begin{method} \hlt{Blending of Models}\\
Integrating multiple alpha signals can be accomplished using different approaches:
\begin{enumerate}[label=\roman*.]
  \setlength{\itemsep}{0pt}
  \item \textbf{Linear Models:}  
  Assume that factors are independent and additive. Multiple regression techniques are typically employed to determine the weight of each factor.

  \item \textbf{Nonlinear Models:}  
  Used when factor relationships are interdependent or evolve over time. Conditional models adjust the weight of one factor based on the performance of another, while rotational models dynamically shift weights according to recent performance metrics.

  \item \textbf{Machine Learning Models:}  
  Although developing machine learning strategies requires substantial effort in data curation, infrastructure, feature engineering, and backtesting, these models can capture micro-level signals that traditional econometric methods may miss. Despite the complexity, ML approaches are becoming more prevalent as computational resources improve.
\end{enumerate}
\end{method}
