\subsection{Overview of Systematic Investments}

The figure below illustrates a live, production-level trading strategy. (Note that the diagram does not include ancillary components—such as research tools—that are also essential for building a complete strategy.)

\begin{figure}[H]
\centering
\input{figures/intro/strategystructure.tikz}
\caption{Live Production Trading Strategy Overview}
\end{figure}

At its core, the trading system is organized into three primary modules:

\begin{enumerate}[label=\roman*.]
\setlength{\itemsep}{0pt}
\item \textbf{Alpha Model:} Predicts the future behavior of the instruments under consideration, thereby generating directional alpha.
\item \textbf{Risk Model:} Limits exposure by controlling risk factors that might not generate returns but could lead to losses (i.e., by setting directional exposure limits on asset classes).
\item \textbf{Transaction Cost Model:} Assesses whether the trading costs required to transition from the current portfolio to a new one are acceptable within the portfolio construction framework.
\end{enumerate}

These models feed into a portfolio construction model that balances profitability and risk to determine the optimal portfolio configuration. This model identifies the necessary trade adjustments, which are then executed by the system.

The execution model receives the required trades and, based on inputs such as trade urgency and current market liquidity, carries out the transactions in an efficient and cost-effective manner.

\begin{method} \hlt{Chains of Production for Alpha Signals}\\
The production process for generating alpha signals consists of the following stages:
\begin{enumerate}[label=\roman*.]
\setlength{\itemsep}{0pt}
\item \textbf{Data Curation:} Involves collecting, cleaning, indexing, storing, adjusting, and delivering data to the production pipeline. This stage requires expertise in market microstructure and data protocols (e.g., FIX).
\item \textbf{Feature Analysis:} Converts raw data into meaningful signals through techniques in information theory, signal extraction, visualization, labeling, weighting, classification, and feature importance assessment. Feature analysts compile and catalogue these insights.
\item \textbf{Strategy Development:} Transforms informative features into actionable investment algorithms. Strategists mine feature libraries for ideas, blending deep financial market knowledge with data science to develop transparent (white-box) strategies—even if some features originate from black-box methods.
\item \textbf{Back-Testing:} Evaluates the investment strategy under various scenarios. This process demands rigorous empirical analysis and includes metadata on how the strategy was formulated.
\item \textbf{Deployment:} Integrates the strategy code into the production system. This step is managed by algorithm specialists and mathematical programmers to ensure fidelity to the prototype while minimizing latency.
\item \textbf{Portfolio Oversight:} Monitors the strategy post-deployment throughout its lifecycle:
  \begin{enumerate}[label=\arabic*.]
  \setlength{\itemsep}{0pt}
  \item \textbf{Embargo:} Initially, the strategy is run on post-backtest data. If performance is consistent with backtesting, the strategy advances to the next phase.
  \item \textbf{Paper Trading:} The strategy is executed on a live feed, simulating real trading conditions. This phase accounts for data parsing, calculation delays, execution lags, and other operational latencies.
  \item \textbf{Graduation:} The strategy is allocated real capital, either individually or as part of an ensemble, with detailed evaluations of risk, returns, and costs.
  \item \textbf{Re-allocation:} Allocations are dynamically adjusted based on live performance. Initial allocations are small and may be increased as the strategy proves itself, with a subsequent reduction if performance declines.
  \item \textbf{Decommission:} If the strategy underperforms over an extended period, it is phased out.
  \end{enumerate}
\end{enumerate}
\end{method}

\subsection{Alpha Models}

A schematic of a live 'production' trading strategy is shown below, but does not include everything else necessary to create the strategy (i.e., research tools).
\begin{figure}[H]
\centering
\includegraphics[scale=0.4]{/fundamentals/strategystructure}
\caption{Live 'production' trading strategy}
\end{figure}
The trading system has three modules:
\begin{enumerate}[label=\roman*.]
\setlength{\itemsep}{0pt}
\item Alpha model: predicts the future of the instruments considered for trading, i.e. directional alpha
\item Risk model: limits amount of exposure to factors that are unlikely to generate returns but could drive losses, i.e. directional exposure limit on an asset class
\item Transaction cost model: determine if the cost of the trades needed to migrate from current portfolio to new portfolio is desirable to the portfolio construction model.
\end{enumerate}
These models feed into a portfolio construction model that balances the tradeoffs of profit and risk to determine the best portfolio to hold. The model finds the differences in trades that need to be executed.\\
The execution model then takes the required trades, and using inputs such as urgency in which the trades need to be executed and dynamics of liquidity in the markets, executes the trades in an efficient and low cost manner.

\subsubsection{Alpha Models}

Theory-driven models tests theories of why markets behave in a manner, and see if they can be used to predict the future. Strategies utilising price-related data are trend and mean reversion; strategies utilising fundamental data are value/yield, growth and quality. Usually more than one model is used in combination.

\begin{definition}
\hlt{(Trend Following)} Markets move in given direction long enough that the trend can be identified. As more data support the bull/bear thesis in an uncertain market, more market participants will adopt the same thesis and hence move the asset price to a new equilibrium.
\end{definition}
An example of a trend is a moving average crossover indicator. This strategy has less than one point of return for every point of downside risk taken, as market behaviour are unstable and episodic.

\begin{definition}
\hlt{(Mean Reversion)} Markets move in opposite direction to the prevailing trend. Short-term imbalances between buyers and sellers due to liquidity forces prices to move abruptly in one direction, which increases probability of trend reversion as liquidity issue is resolved.
\end{definition}
An example of mean reversion strategy is statistical arbitrage, which bets on convergence of prices of similar stocks whose prices have diverged.\\

Trend and mean reversion strategies are not at odds. Longer-term trends can occur despite smaller oscillations around these trends occurring in the shorter term, hence both strategies may be use din conjunction.\\

\begin{definition}
\hlt{(Value/Yield)} Value strategies are usually ratios of some fundamental factor against the price of the instrument, inverted to keep the ratio consistent. The higher the yield, the cheaper the instrument.
\end{definition}
Market tend to overestimate risk in risky instruments and underestimate the risk in less risky instruments. When the strategy is executed on a relative basis, i.e., buying the undervalued security and selling the overvalued one against it, this is a \hlt{carry trade}. The difference between yield received and yield paid is the \hlt{carry}.\\

\hlt{Quant Long Short (QLS)} ranks stocks by attractiveness based on various factors such as value, then buy the higher-ranked stocks while shorting the lower-ranked stocks.

\begin{definition}
\hlt{(Growth)} Make predictions based on asset's expected or historically observed level of economic growth. Forward-looking growth expectations are typically used as a metric.
\end{definition}
Growth is trending, and strongest growers are becoming more dominant relative to competitors. Macro growth factors may be used on foreign exchange, while micro growth factors may be used on companies.

\begin{definition}
\hlt{(Quality)} All else being equal, it is better to long high quality and short low quality. Capital safety is important. Factors include earnings quality, equity-to-debt ratios etc.
\end{definition}

Data-driven models are more difficult to understand, with more complicated mathematics. Relies on data mining, more technically challenging and far less widely practiced. Typically more used in high-frequency space, as they can discern how market behaves without caring about the economic theory or rational.

\subsubsection{Strategy Implementation}

An implementation approach requires a forecast target, time horizon, bet structure, investment universe, model specification, and run frequency.\\

\hlt{(Forecast Target)} Models may forecast direction, magnitude, duration of move, and may include probability into the forecast. Signal strength is of importance, defined by a larger expected return and/or higher likelihood of return. A higher level of signal strength results in a bigger bet taken on the position.\\

\hlt{(Time Horizon)} Models may have forecast horizons ranging from microseconds to years. There are more variability between short-term and long-term strategies, as short-term strategies are making very large number of trades compared to long-term strategies.\\

\hlt{(Bet Structure)} Models can be made to forecast an instrument relative in itself or to others. For relative forecasts, smaller clusters (pairs) or larger clusters (sectors) may be used. For pairs, few assets can be compared precisely and directly. Large cluster grouping may eliminate impact of general movement of the sector and hence focus on the relative movement of stocks within the sector, allowing for clearer distinction between group behaviour and idiosyncratic behaviour. Clusters may be created either via statistical methods or using heuristics (i.e., fundamentally defined industry groups).\\
Statistical methods may be fooled by data, leading to bad grouping. Heuristic grouping may be imprecise for conglomerates, and may be too rigid. Relative alpha strategies tend to exhibit smoother returns during normal times than intrinsic alpha strategies, but may face incorrect groupings during stressful periods. This may be mitigated by utilising several grouping techniques in concert.\\

\hlt{(Investment Universe)} Choices made on geography, asset class, instrument class, and exclusions. Generally, liquidity is preferred so estimations of transaction costs are reliable. Large quantities of high quality data is required, which is found in highly liquid and developed markets. Instruments with consistent behaviour is preferred, hence biotech stocks are excluded due to sudden, violent price changes. Hence, the most common asset classes and instruments modelled are common stocks, futures (on bonds and equity indices) and forex.\\

\hlt{(Model Specification)} Focuses on definition of the strategy mathematically, and may be the source of alpha. Specification details in terms of machine learning or data mining techniques are also defined, to assist in fitting models to the data and setting parameter values. Refitting frequency is also defined to refresh the model and make it adapt to current market conditions; may lead to greater risk of overfitting.\\

\hlt{(Run Frequency)} Run frequency of model is defined, from monthly to real time frequency. Increasing frequency of runs lead to greater number of transactions and hence higher transaction costs, and risk of moving portfolio based on noisy data. Less frequency of runs lead to smaller number of larger-sized trades, hence may move the market with block trades; may also miss opportunities to trade at more favourable prices.

\subsubsection{Blending Models}

The three most common approaches are linear models, nonlinear models, and machine learning models. If models are not combined, then several portfolios are constructed based on output from each model, then combined using portfolio construction techniques. The best method depends on the model.

Linear models require independence of factors, and each factor to be additive. To determine the weight of each alpha factor, multiple regression techniques may be used.

Nonlinear models are used when factors are not independent, or the relationship changes over time. Conditional models base the weight of one factor on the reading of another factor. Rotational models assign weights of factors that fluctuate over time based on updated calculations of the various signal's weights, giving higher weights to factors with better performance recently.

Machine learning models applied to mixing alpha factors are more successful than the approach being used to forecast markets. For rotational models, many approaches to mixing alpha factors periodically update optimal weights based on the changing and growing dataset.



\subsubsection{Risk Models}

Risk model concerns the intentional selection and sizing of exposures to improve the quality and consistency of returns. By pursing an alpha, we want to be invested in the movement of the exposure to profit in the long run.

\begin{method} \hlt{Limiting Size of Risk}\\
The quantitative risk models that limit the size of risk varies by the manner in which size is limited, how risk is measured, and what is having its size limited.\\
Size limits can be limited by hard constraints and penalties. A hard limit may be arbitrary, hence penalty functions may be built to allow a position to increase beyond the limit level, only if the alpha model expects a significantly larger return. The levels of limits and penalties may be determined from either theory or data.\\
To measure risk, there are two methodologies. The first is longitudinal, and measures risk through the volatility of an instrument. The second is to measure the correlation or covariance between assets (dispersion).\\
Size limiting may be applied to single positions and groups of positions (sectors, asset classes). It may also be applied to various types of risks and the amount of portfolio leverage.
\end{method}

\begin{method} \hlt{Limiting the Types of Risk}\\
To eliminate unintentional exposure as there is no expectation of being compensated sufficiently for accepting them. This can be achieved through theoretical or empirical risk models.
\begin{enumerate}[label=\roman*.]
\setlength{\itemsep}{0pt}
\item Theory-Driven Risk Models: focuses on systematic risk factors, derived from economic theory. Systematic risks cannot be diversified away. Equity may have market risk, sector risk, market capitalisation risk etc. Fixed income may have interest rate risk.
\item Empirical Risk Models: uses historical data to determine the unnamed systematic risks that should be measured and mitigated. Uses principal component analysis (PCA) to discern unnamed systematic risks that may correspond to named risk factors. Used by statistical arbitrage traders who are betting on exactly the component of an asset's return not explained by systematic risks.
\end{enumerate}
\end{method}

\subsection{Transaction Cost Models}

Trade is made only if it increases the odds or magnitude of return (from alpha model), or if it decreases the odds or magnitudes of loss (from risk model). However, this improvement should be higher than cost of trading. The transaction cost model is not designed to minimise cost of trading, only to inform portfolio construction engine the cost of making any given trade.

\subsubsection{Transaction Costs}

Transaction costs have three major components: commissions and fees, slippage, market impact.\\

\hlt{(Commissions and Fees)} Paid to brokerages (access to other market participants), exchanges (improved transaction security) and regulators (operational infrastructure) for the services provided. The bank's infrastructure is used by quants, where the brokerage commissions are rather small on a per-trade basis.\\
Brokers also collect clearing and settlement fees. Clearing is the activity involving regulatory reporting and monitoring, tax handling, and handling failure, taken place in advance of settlement. Settlement is the delivery of securities in exchange for payment in full.\\

\hlt{(Slippage)} The change in price between the time the quant system decides to transact and the time when the order is at the exchange for execution. Trend-following strategies suffer most from slippage as assets are already moving in desired direction; mean-reverting strategies suffer the least from slippage. The lower the latency to market, the smaller the slippage. The more volatile an asset, the bigger the slippage.\\

\hlt{(Market Impact)} Measures how much an order moves the market by its demand for liquidity. The impact of the trade on the market is unknown until the trade has already been completed. There may also be interaction between slippage and market impact (i.e., selling when a stock is trending upwards).

\subsubsection{Types of Models}

The four basic types of transaction cost models are flat, linear, piece-wise linear, and quadratic.\\

\hlt{(Flat Model)} Cost of trading is the same, regardless of size of order. Model is reasonable if size traded is nearly always about the same, and liquidity remains sufficiently constant.\\

\hlt{(Linear Model)} Cost of trading increases at a constant rate relative to size of order. Better estimate than flat transaction cost model.\\

\hlt{(Piece-Wise Linear Model)} Using piece-wise linear functions to model costs. Balance between simplicity and accuracy; better accuracy than flat or linear models.\\

\hlt{(Quadratic Model)} Most computationally intensive, but also most accurate.
\subsubsection{Portfolio Construction Models}

Portfolio construction models generally come in two major forms: rule-based models and optimisers. Rule-based models are built on heuristics—they can be very simple or quite complex—and are often derived from human experience and trial-and-error. In contrast, optimiser-based models rely on an objective function and use algorithmic methods to achieve the desired portfolio outcome.

\begin{definition} \hlt{Rule-Based Models}
\begin{enumerate}[label=\roman*.]
\setlength{\itemsep}{0pt}
\item \textbf{Equal Position Weighting:} Applied when portfolio manager believes that once a position is deemed good enough to own, no further information is needed to decide its size. Strength of signal does not influence the weighting. Model assumes there is enough statistical power to predict not only direction but also magnitude relative to other forecasts within the portfolio. As a result, portfolio may place a few large bets on the strongest forecast and many smaller bets on less dramatic signals; however, this can lead to taking excessive risk in an idiosyncratic event on an apparently attractive position, which may cause adverse selection bias.
\item \textbf{Equal Risk Weighting:} Strategy adjusts position sizes inversely to their volatility or another measure of risk. More volatile positions has smaller allocations, whereas less volatile positions has more allocation. Because the unit of risk is typically a backward-looking measure, such as historical volatility, if volatility shifts over time, the model might be misled.
\item \textbf{Alpha-Driven Weighting:} Position size is primarily determined by alpha model. Alpha signal guides size of position, usually subject to predetermined size limits. Additional constraints often include limits on the total bet size within a group. A function may also be used to relate the forecast's magnitude to the position size. In futures trend following, it might suffer from sharp drawdowns, as it heavily relies on the accuracy of the alpha signals.
\item \textbf{Decision-Tree Weighting:} A decision-making process is used to determine the allocation for each instrument based on both the type of alpha model and the type of instrument. Constraints, such as percentage limits for allocation, may also be imposed. However, as more alpha models or instrument types are introduced, the decision tree can grow significantly in complexity.
\end{enumerate}

\end{definition}

\begin{remark} \hlt{Optimisers Models Parameters}\\
The pioneering model in optimiser-based portfolio construction is Harry Markowitz's mean variance optimisation (MVO), which is founded on the principles of modern portfolio theory (MPT). The main inputs to these models include the expected return (mean), asset variance, and the expected correlation matrix. Other inputs typically involve the portfolio’s size in currency terms, the desired risk level (such as volatility or expected drawdown), and additional constraints like liquidity and universe limits. Model uses an objective function paired with an algorithm that seeks to maximise the portfolio’s return relative to its volatility.
\begin{enumerate}[label=\roman*.]
\setlength{\itemsep}{0pt}
\item \textbf{Expected Return:} Derived from alpha models, this input captures not only the direction but also the magnitude of the expected returns.
\item \textbf{Expected Volatility:} Typically estimated using stochastic volatility forecasting methods (i.e., GARCH), this input accounts for periods of high and low volatility along with occasional jumps.
\item \textbf{Expected Correlation:} Given that instrument correlations can fluctuate over time, it is often more effective to group similar assets together before calculating the correlations within each group.
\end{enumerate}
\end{remark}

\begin{method} \hlt{Optimisation Techniques}
\begin{enumerate}[label=\roman*.]
\setlength{\itemsep}{0pt}
\item \textbf{Unconstrained Optimisation:} Most basic form of optimisation with no constraints applied. Might result in a portfolio that invests all available capital in a single instrument—specifically, the one with the highest risk-adjusted return.
\item \textbf{Constrained Optimisation:} Constraints such as position limits or limits on groups of instruments are applied. These constraints can sometimes have a greater influence on the portfolio construction than the optimiser itself.
\item \textbf{Black-Litterman Optimisation:} This combines investor forecasts with a measure of confidence in those forecasts, blending them with historical data. It adjusts the historically observed correlation levels by incorporating the investor’s return expectations for various instruments.
\item \textbf{Grinold and Kahn's Approach:} Instead of directly sizing positions, this constructs a portfolio of signals. It creates factor portfolios, each typically based on a single type of alpha forecast. After backtesting these factor portfolios, their return series are then treated as instruments for the optimiser. Since the number of factor portfolios is usually limited (typically no more than 20), the optimisation problem becomes more manageable. This method also allows for the inclusion of a risk model, transaction cost model, overall portfolio size, and risk targets as additional inputs.
\item \textbf{Resampled Efficiency:} Seeks to improve the input parameters for optimisation by reducing the oversensitivity to estimation errors. It does so by employing Monte Carlo simulation to resample data, thereby mitigating the estimation error in the inputs.
\item \textbf{Data-Mining Approaches:} Leverage machine learning techniques—such as supervised learning or genetic algorithms—to explore a wide range of potential portfolios and identify the optimal one.
\end{enumerate}
\end{method}
\subsubsection{Execution Model}

There are two primary methods to execute a trade: electronically or via a human intermediary. In electronic execution, direct market access (DMA) is employed, allowing traders to leverage the brokerage firm’s infrastructure and exchange connectivity to trade directly on electronic markets.\\
Execution algorithms can be developed in-house, sourced from brokers, or obtained from third-party vendors.\\
Brokerages also provide portfolio bidding services. In these arrangements, the “blind” portfolio for the transaction is characterized by features such as valuation ratios of long and short positions, sector allocation, market capitalization, and similar metrics. The broker then quotes a fee—expressed in basis points relative to the gross market value of the portfolio traded—which offers the trader a measure of certainty. Once an agreement is reached, the broker collects the fee and assumes the risk of executing the portfolio at future market prices, which may turn out to be more or less favorable than the initially guaranteed prices..

\begin{remark} \hlt{Order Execution Algorithm Parameters}
\begin{enumerate}[label=\roman*.] \setlength{\itemsep}{0pt}

\item \textbf{Aggressive vs Passive:} Algorithm decides whether to use an aggressive or passive order based on how immediately the trade must be executed. Market orders are inherently aggressive. Limit order placed at current best quote is relatively aggressive; limit order positioned below current bid is considered passive.\\
Many exchanges reward liquidity providers for placing passive orders, while charging traders for consuming liquidity. When an order crosses the spread, it effectively uses liquidity from another trader’s passive order, thereby reducing available liquidity. In return, if the passive order is executed, the trader can benefit from a better transaction price along with a commission rebate from the exchange.\\
Momentum-based strategies typically favour aggressive orders, whereas mean reversion strategies lean towards passive orders. A strong, reliable signal justifies a more aggressive execution, while a weaker or less certain signal may call for a more passive approach. An intermediate strategy might involve placing limit orders between the best current bid and offer.
\item \textbf{Large vs Small Order:} Large order may be divided into several smaller orders to reduce market impact, though this carries risk of adverse price movements. The optimal size of each order segment is determined by estimates from a transaction cost model and an analysis of the appropriate level of aggressiveness.
\item \textbf{Hidden vs Visible Order:} Visible order discloses certain trading intentions, whereas a hidden order conceals this information, helping to prevent market imbalances; however, hidden orders typically suffer from lower execution priority. In algorithmic trading, the practice of using hidden orders—where a large order is segmented into many smaller parts, with most being placed as hidden—is known as “iceberging.”
\item \textbf{Order Routing:} When multiple liquidity pools exist for same instrument, smart order routing systems are used to determine most suitable pool for executing a given order. These systems evaluate factors such as depth of liquidity on various electronic communication networks (ECNs) and connectivity speeds.
\item \textbf{Cancelling and Replacing Orders:} Traders may submit a large number of orders without the expectation of execution, only to rapidly cancel and replace them. This practice helps gauge the market’s response to changes in order book depth, offering insights into potential profitable patterns. For example, if a trader intends to buy a significant number of shares, they might place numerous small orders at prices further from the market and then cancel them, thereby improving market perception.
\end{enumerate}
\end{remark}

\begin{definition} \hlt{High Frequency Trading}\\
High Frequency Trading (HFT) involves alpha-driven strategies that focus on extremely short-term bets—often executed in seconds or less—referred to as \hlt{microstructure alphas}. These strategies analyze liquidity patterns in the order book. Larger quantitative groups may also incorporate these insights into their execution models to reduce the costs associated with entering trades. Even marginal improvements per trade can accumulate significantly over time. However, pursuing microstructure alpha as a standalone high frequency strategy necessitates substantial investments in both infrastructure and research.\\
Machine learning techniques may be applied to detect patterns in how other market participants execute orders. When competitors’ execution models are less refined, their patterns become more apparent, allowing an ML strategy to exploit these patterns in the future. Short-term patterns tend to exhibit a degree of stability.
\end{definition}

\begin{definition} \hlt{HFT Shark Strategy}\\
This is designed to detect large orders that have been fragmented (or “iceberged”) by observing the rapid filling of a series of very small trades. Quick execution of these small orders can indicate the presence of a large hidden order. The strategy then involves front-running this iceberg order by placing visible trades ahead of it. Consequently, the iceberg order is forced to move market prices upward in order to execute its trades. Once the iceberg order is fully executed, the resulting favorable price movement enables the shark to exit its position quickly, thereby securing a relatively risk-free profit.
\end{definition}

\begin{remark} \hlt{HFT Trading Infrastructure}\\
By using a broker that acts as a trading agent, traders can offload infrastructure requirements and avoid dealing directly with regulatory and other operational constraints.
High frequency strategies may also employ colocation or sponsored access. In a colocation setup, traders place their trading servers in close physical proximity to the exchange to minimize latency.\\
The Financial Information eXchange (FIX) protocol is the standard for real-time electronic communication among market participants. Although open source FIX engines are available, high frequency traders often develop their own proprietary FIX engines to ensure optimal execution speeds.
\end{remark}


\subsection{Research}

The scientific method is as follows: 
\begin{enumerate}[label=\arabic*.]
\setlength{\itemsep}{0pt}
\item Researcher observe a phenomenon in the market and construct a theory.
\item Researcher seeks out information to test the theory.
\item Researcher tests the theory, and with enough confidence, risk some capital on the validity of the theory.
\end{enumerate}

Idea generation comes from four sources:
\begin{enumerate}[label=\arabic*.]
\setlength{\itemsep}{0pt}
\item Observing the market, using the scientific method to test the theory
\item Academic literature, requiring significant time to read academic journals, working papers, and conference presentations for ideas. Literature from other fields such as astronomy, physics, or psychology, may provide ideas relevant to quant finance problems.
\item Migration of a researcher or portfolio manager from one quant shop to another.
\item Lessons from activities of discretionary traders
\end{enumerate}

\subsubsection{Model Testing}

\hlt{(In-Sample Training)} Train a model by finding optimal parameters over an in-sample period. The sample for fitting the model must be chosen in terms of appropriate length and breadth.\\

\hlt{(Model Quality)} A model can be assessed on the following fronts:
\begin{enumerate}[label=\roman*.]
\setlength{\itemsep}{0pt}
\item Cumulative profit graph: if profit profile is not smooth, with long periods of inactivity, sharp losses and gains, then the model may have issues
\item Average annual rate of return: indicates how well the strategy made on historical data
\item Variability of returns: the less variable the level of returns, the better the strategy. May look at lumpiness of returns, which is the portion of strategy's total returns that comes from periods that are significantly above average (measures consistency of returns).
\item Worse Peak-to-Valley Drawdowns: measures maximum decline from any cumulative peak in profit curve. The lower the drawdown the better the strategy. Also, to measure recovery period after drawdowns; the shorter the recovery period the better the strategy.
\item Predictive Power: R-squared statistic may be used, which shows how much of the variability of the predicted asset have been accounted for. A exceedingly high $R^2$ in would be 0.05 out of sample. Instrument returns may be bucketed by deciles; a model with reliable predictive power is one that appropriately buckets the instruments correctly.
\item Percentage Winning Trades, Winning Time Periods: whether the strategy tends to make profits from a small portion of trades that do very well, or from a large number of trades. 
\item Ratios of Returns vs Risk: Statistics such as risk-adjusted return, Sharpe ratio, information ratio, Sterling ratio, Calmer ratio, Omega ratio.
\item Relationship with Other Strategies: value-add of new strategy compared with results of existing strategy with and without the new idea.
\item Time decay: understand strategy returns if trades are initiated on lagged basis after receiving a trading signal. Determine strategy sensitivity to timeliness with information received, and crowdedness of strategy.
\item Sensitivity to specific parameters: high quality strategy has small changes in outcomes from slight changes in parameters. Or else this may be a sign that model may be overfitted.
\item Overfitting: plot a graph of parameter value vs function outcome; a good model has a flatter curve with no jumps.  Models that are parsimonious (less parameters) uses less assumptions, hence less overfitting.
\end{enumerate}

\hlt{(Out-of-Sample Testing)} Tests if model works in real-life. $R^2$ is typically used to test robustness of model. If out-of-sample $R^2$ is close to in-sample $R^2$, then the strategy is good. Rolling out-of-sample technique may be used to refresh the model over time. Look-ahead bias may be avoided by separating strategy research function from strategy selection function, and withholding a significant portion of database from researchers.\\

\hlt{(Assumptions of Trading)} Overestimation of trading costs may cause portfolio to hold positions for longer than optimal, and underestimation may result in high portfolio turnover and bleed from trading costs. Assumptions on availability of short positions must also be made; hard-to-borrow lists must be taken into consideration.
\subsubsection{Risk Assessment}

\begin{definition} \hlt{Model Risks}\\
Quantitative models inherently carry model risk—the risk that a model fails to accurately describe, match, or predict real-world phenomena. Each element of a quant model is subject to its own potential for error.
\begin{enumerate}[label=\roman*.]
\setlength{\itemsep}{0pt}
  \item \textbf{Inapplicability of Modelling:} This risk arises when a quant model is applied to a problem for which it was not designed, or when a particular technique is misapplied to a given scenario.
  \item \textbf{Model Misspecification:} This occurs when the model does not properly reflect the real world. Although a model may perform adequately under normal conditions, it can fail under extreme circumstances.
  \item \textbf{Implementation Errors:} These include mistakes in coding or system design. Additionally, errors can occur if the model components are executed in an incorrect sequence.
\end{enumerate}
\end{definition}

\begin{definition} \hlt{Regime Change Risk}\\
Quant models are built on historical data, relying on relationships that have prevailed over time. When a regime change occurs, these historical relationships and behaviors can shift, leading the model to lose its effectiveness.
\end{definition}

\begin{definition} \hlt{Exogenous Shock Risk}\\	
This type of risk is driven by external events that are not inherent to the market itself, such as terrorist attacks, the outbreak of wars, bank bailouts, or regulatory changes (e.g., modifications to shorting rules). In such cases, discretionary overrides may be necessary.
\end{definition}

\begin{definition} \hlt{Contagion Risk}\\
Contagion risk occurs when multiple investors follow similar strategies. There are two components to this risk: first, the extent to which a quant strategy is crowded; and second, the additional positions held by other investors that might force them to exit the strategy in a panic (often referred to as the ATM effect).\\
Quant liquidation crisis may be triggered by factors such as the sheer size and popularity of quantitative strategies, suboptimal returns by operators leading up to a crisis, the cross-collateralisation of many strategies within funds, and risk targeting—where risk managers aim to maintain a specific level of volatility across their funds or strategies.
\end{definition}

\begin{method} \hlt{Risk Monitoring Methods}
\begin{enumerate}[label=\roman*.]
\setlength{\itemsep}{0pt}
  \item \textbf{Exposure Monitoring Tools:} These tools aggregate current positions by various exposures (e.g., valuation, momentum, volatility) to monitor both gross and net exposures across sectors, industries, market capitalisation buckets, and style factors.
  \item \textbf{Profit and Loss Monitors:} By comparing the current portfolio against the previous day’s closing prices, these monitors utilize intraday performance charts and also examine the source of profits and the hit rate (i.e., the percentage of positions where the strategy is profitable).
  \item \textbf{Execution Monitors:} These displays track the status of orders—indicating which are being processed and which have been completed—along with details such as transaction sizes and prices. They also measure fill rates for limit orders in passive strategies, and monitor slippage and market impact.
  \item \textbf{System Performance Monitors:} These monitors are responsible for detecting software or infrastructure errors, assessing CPU performance, measuring the speed of various stages of automated processes, and tracking communication latency.
\end{enumerate}
\end{method}
