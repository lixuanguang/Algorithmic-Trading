\subsubsection{Transaction Cost Models}

A trade is executed only if it improves the probability or magnitude of a return (as predicted by alpha model) or reduces likelihood or extent of a loss (as determined by risk model). However, this improvement must be greater than cost of trading. Note that transaction cost model is intended not to minimize trading costs directly, but rather to inform the portfolio construction engine of the costs associated with executing a given trade.

\begin{remark} \hlt{Transaction Cost Components}
\begin{enumerate}[label=\roman*.]
\setlength{\itemsep}{0pt}
    \item \textbf{Commissions and Fees:} These are payments made to brokerages (for accessing other market participants), exchanges (for enhanced transaction security), and regulators (for maintaining operational infrastructure). In the context of quantitative trading, where bank infrastructure is utilized, brokerage commissions tend to be minimal on a per-trade basis.\\[1mm]
    Brokers also charge clearing and settlement fees. \emph{Clearing} involves activities such as regulatory reporting and monitoring, tax handling, and failure management, all of which occur before settlement. \emph{Settlement} is the final delivery of securities in exchange for full payment.
    \item \textbf{Slippage:} This refers to the change in price between the moment the trading system decides to execute a transaction and the time the order is actually sent to the exchange. Trend-following strategies tend to experience more slippage because the assets are already moving in the desired direction, whereas mean-reverting strategies are less affected. Reduced latency to market minimizes slippage, while higher asset volatility increases it.  
    \item \textbf{Market Impact:} This measures the extent to which an order influences the market through its demand for liquidity. The market impact remains uncertain until after the trade is executed. Additionally, there can be an interaction between slippage and market impact (i.e., selling during an upward trending market).
\end{enumerate}
\end{remark}

\begin{definition} \hlt{Types of Transaction Cost Models}
\begin{enumerate}[label=\roman*.]
\setlength{\itemsep}{0pt}
    \item \textbf{Flat Model:} Assumes that the cost of trading remains constant regardless of the order size. This model is appropriate when the traded size is nearly uniform and liquidity remains stable.
    \item \textbf{Linear Model:} In this model, the trading cost increases at a constant rate relative to the order size. It provides a better estimate than the flat model.  
    \item \textbf{Piece-Wise Linear Model:} This approach employs piece-wise linear functions to model costs. It strikes a balance between simplicity and accuracy, offering improved precision compared to flat or linear models.
    \item \textbf{Quadratic Model:} Although the most computationally intensive, the quadratic model delivers the highest accuracy in cost estimation.
\end{enumerate}
\end{definition}
