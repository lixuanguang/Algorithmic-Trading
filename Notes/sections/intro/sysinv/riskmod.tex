\subsubsection{Risk Models}

Risk models are indispensable tools in algorithmic trading, providing a quantitative framework to assess, monitor, and manage the inherent risks of financial markets. They not only serve to measure exposure but also guide portfolio construction, hedging strategies, and overall risk control.

\begin{method} \hlt{Factor-Based Models}\\
Factor-based models decompose asset returns into contributions from systematic factors and idiosyncratic components. The most common factors include:
\begin{enumerate}[label=\roman*.]
	\setlength{\itemsep}{0pt}
    \item \textbf{Market Factor:} Captures the overall movement of the market.
    \item \textbf{Size Factor:} Reflects the differential risk associated with companies of varying market capitalizations.
    \item \textbf{Value Factor:} Accounts for risk due to discrepancies between market prices and fundamental valuations.
    \item \textbf{Momentum Factor:} Measures the tendency of asset prices to continue in their current trajectory.
\end{enumerate}
This allows traders to understand which elements drive portfolio risk and adjust exposures accordingly.
\end{method}

\begin{method} \hlt{Statistical Models}\\
Statistical risk models leverage historical data and probabilistic techniques to quantify risk parameters.
\begin{enumerate}[label=\roman*.]
	\setlength{\itemsep}{0pt}
    \item \textbf{Historical Simulation:} Directly computing risk metrics from past return distributions.
    \item \textbf{Monte Carlo Simulation:} Generating a large number of potential future return scenarios to estimate risk under diverse conditions.
    \item \textbf{Parametric Methods:} Employing analytical formulas based on assumed return distributions to calculate key risk measures.
\end{enumerate}
These are useful for dynamically updating risk assessments as new market data become available.
\end{method}

\begin{method} \hlt{Limiting Size of Risk}\\
Quantitative risk models are designed to limit the size of exposures to enhance return consistency.
\begin{enumerate}[label=\roman*.]
	\setlength{\itemsep}{0pt}
    \item \textbf{Constraint Mechanisms:}
        \begin{enumerate}[label=\alph*.]
            \item \textbf{Hard Constraints:} Absolute limits imposed on position sizes, which may be set arbitrarily.
            \item \textbf{Penalty Functions:} Flexible constraints where positions can exceed the limit if the alpha model forecasts significantly higher returns, with penalties applied for surpassing the prescribed levels.
        \end{enumerate}
    \item \textbf{Risk Measurement Approaches:}
        \begin{enumerate}[label=\alph*.]
            \item \textbf{Longitudinal Analysis:} Evaluates risk by assessing the volatility of an instrument over time.
            \item \textbf{Dispersion Analysis:} Measures risk by analyzing the correlation or covariance between assets.
        \end{enumerate}
\end{enumerate}
These methods can be applied to single positions, groups of positions (such as sectors or asset classes), different types of risk exposures, and overall portfolio leverage.
\end{method}

\begin{method} \hlt{Limiting the Types of Risk}\\
To eliminate unintentional exposure as there is no expectation of being compensated sufficiently for accepting them. This can be achieved the following measures:
\begin{enumerate}[label=\roman*.]
	\setlength{\itemsep}{0pt}
    \item \textbf{Market Exposure Restriction:} Focus on specific market segments to avoid undue exposure to volatile or unpredictable markets.
    \item \textbf{Leverage Management:} Limit the use of leverage by enforcing conservative leverage ratios to mitigate amplified losses.
    \item \textbf{Stop-Loss Policies:} Implement strict stop-loss rules to automatically exit positions that breach predetermined loss thresholds.
    \item \textbf{Position Size Controls:} Cap individual position sizes relative to overall portfolio risk, ensuring no single trade unduly influences portfolio performance.
    \item \textbf{Regular Risk Assessments:} Continuously monitor risk metrics and adjust risk parameters to reflect evolving market conditions.
\end{enumerate}
\end{method}
