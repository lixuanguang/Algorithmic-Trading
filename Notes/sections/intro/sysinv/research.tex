\subsubsection{Research}

\begin{definition} \hlt{Scientific Method}
\begin{enumerate}[label=\arabic*.]
\setlength{\itemsep}{0pt}
\item Researcher observe a phenomenon in the market and construct a theory.
\item Researcher seeks out information to test the theory.
\item Researcher tests the theory, and with enough confidence, risk some capital on the validity of the theory.
\end{enumerate}
\end{definition}

\begin{remark} \hlt{Sources of Alpha Idea Generation}
\begin{enumerate}[label=\arabic*.]
\setlength{\itemsep}{0pt}
\item Observing the market, using the scientific method to test the theory
\item Academic literature, requiring significant time to read academic journals, working papers, and conference presentations for ideas. Literature from other fields such as astronomy, physics, or psychology, may provide ideas relevant to quant finance problems.
\item Migration of a researcher or portfolio manager from one quant shop to another.
\item Lessons from activities of discretionary traders
\end{enumerate}
\end{remark}

\begin{remark} \hlt{Model Quality Assessment}
\begin{enumerate}[label=\roman*.]
\setlength{\itemsep}{0pt}
\item \textbf{Cumulative Profit Graph:} A smooth profit curve is ideal; if the graph shows long periods of inactivity or exhibits sharp, erratic losses and gains, it may signal issues with the model.
\item \textbf{Average Annual Rate of Return:} Indicates the historical performance level of the strategy.
\item \textbf{Variability of Returns:} Lower variability in returns is preferable, as it suggests consistency. Examining the "lumpiness"—the share of total returns derived from periods significantly above average—can further measure return consistency.
\item \textbf{Peak-to-Valley Drawdowns:} Measures maximum decline from any cumulative peak in the profit curve. Lower drawdowns, along with shorter recovery periods after drawdowns, reflect a more robust strategy.
\item \textbf{Predictive Power:} The R-squared statistic can be employed to assess how much of the variability in the predicted asset is explained by the model. For example, an exceptionally high \( R^2 \) (around 0.05 out of sample) may warrant further scrutiny. Additionally, bucketing instrument returns by deciles can help verify whether the model categorizes them accurately.
\item \textbf{Percentage of Winning Trades and Winning Time Periods:} Determines whether the strategy relies on a small number of highly profitable trades or on a larger volume of moderately successful trades.
\item \textbf{Risk-Adjusted Return Ratios:} Evaluate statistics such as the Sharpe ratio, information ratio, Sterling ratio, Calmar ratio, and Omega ratio to assess the balance between returns and risk.
\item \textbf{Relationship with Other Strategies:} Consider the incremental value provided by a new strategy by comparing its performance with existing strategies, both independently and in combination.
\item \textbf{Time Decay:} Examine how the returns of the strategy are affected if trades are initiated on a lagged basis after receiving a trading signal. This helps to determine the sensitivity of the strategy to the timeliness of information and the degree of market saturation.
\item \textbf{Sensitivity to Specific Parameters:} A high-quality strategy should show only minor variations in outcomes with small changes in its parameters; large fluctuations may indicate overfitting.
\item \textbf{Overfitting:} By plotting parameter values against the corresponding outcomes, one should observe a relatively flat curve with no abrupt jumps. Models that are parsimonious—that is, those that rely on fewer parameters—tend to be less prone to overfitting.
\end{enumerate}
\end{remark}

\begin{remark} \hlt{Other Considerations in Model Testing}\\
It is crucial to note that overestimating trading costs can lead to holding positions longer than optimal, whereas underestimating these costs might result in excessive portfolio turnover and a detrimental bleed from trading expenses. Moreover, assumptions regarding the availability of short positions must be carefully considered, especially with respect to hard-to-borrow lists.
\end{remark}
