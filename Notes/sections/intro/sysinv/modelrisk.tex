\subsubsection{Risk Assessment}

\begin{definition} \hlt{Model Risks}\\
Quantitative models inherently carry model risk—the risk that a model fails to accurately describe, match, or predict real-world phenomena. Each element of a quant model is subject to its own potential for error.
\begin{enumerate}[label=\roman*.]
\setlength{\itemsep}{0pt}
  \item \textbf{Inapplicability of Modelling:} This risk arises when a quant model is applied to a problem for which it was not designed, or when a particular technique is misapplied to a given scenario.
  \item \textbf{Model Misspecification:} This occurs when the model does not properly reflect the real world. Although a model may perform adequately under normal conditions, it can fail under extreme circumstances.
  \item \textbf{Implementation Errors:} These include mistakes in coding or system design. Additionally, errors can occur if the model components are executed in an incorrect sequence.
\end{enumerate}
\end{definition}

\begin{definition} \hlt{Regime Change Risk}\\
Quant models are built on historical data, relying on relationships that have prevailed over time. When a regime change occurs, these historical relationships and behaviors can shift, leading the model to lose its effectiveness.
\end{definition}

\begin{definition} \hlt{Exogenous Shock Risk}\\	
This type of risk is driven by external events that are not inherent to the market itself, such as terrorist attacks, the outbreak of wars, bank bailouts, or regulatory changes (e.g., modifications to shorting rules). In such cases, discretionary overrides may be necessary.
\end{definition}

\begin{definition} \hlt{Contagion Risk}\\
Contagion risk occurs when multiple investors follow similar strategies. There are two components to this risk: first, the extent to which a quant strategy is crowded; and second, the additional positions held by other investors that might force them to exit the strategy in a panic (often referred to as the ATM effect).\\
Quant liquidation crisis may be triggered by factors such as the sheer size and popularity of quantitative strategies, suboptimal returns by operators leading up to a crisis, the cross-collateralisation of many strategies within funds, and risk targeting—where risk managers aim to maintain a specific level of volatility across their funds or strategies.
\end{definition}

\begin{method} \hlt{Risk Monitoring Methods}
\begin{enumerate}[label=\roman*.]
\setlength{\itemsep}{0pt}
  \item \textbf{Exposure Monitoring Tools:} These tools aggregate current positions by various exposures (e.g., valuation, momentum, volatility) to monitor both gross and net exposures across sectors, industries, market capitalisation buckets, and style factors.
  \item \textbf{Profit and Loss Monitors:} By comparing the current portfolio against the previous day’s closing prices, these monitors utilize intraday performance charts and also examine the source of profits and the hit rate (i.e., the percentage of positions where the strategy is profitable).
  \item \textbf{Execution Monitors:} These displays track the status of orders—indicating which are being processed and which have been completed—along with details such as transaction sizes and prices. They also measure fill rates for limit orders in passive strategies, and monitor slippage and market impact.
  \item \textbf{System Performance Monitors:} These monitors are responsible for detecting software or infrastructure errors, assessing CPU performance, measuring the speed of various stages of automated processes, and tracking communication latency.
\end{enumerate}
\end{method}